\chapter{Introducción}
\label{cap:Introduccion}

Introducción al tema, entorno y justificación de la importancia del trabajo abordado.

Estructura del resto del documento.

% ------------------------------------------------------------------------------
% Ejemplos para la plantilla
% ------------------------------------------------------------------------------
\section{Ejemplos de listas}
\label{sec:ejListas}
\index{ejemplos} % Véase cómo se incluyen entradas en el índice alfabético
A continuación se van a añadir algunos ejemplos que pueden emplearse al redactar la memoria.

\index{ejemplos!listas} % Para el índice
\noindent Ejemplo de lista con \emph{bullet} especial. 
% Ejemplo: Lista con bullets especiales
% ============
\begin{itemize}
	\item[*] peras
	\item manzanas
	\item[\ding{170}] naranjas
\end{itemize}

\noindent Ejemplo de lista compacta (también se puede emplear el entorno para enumeraciones \emph{compactenum})
% Ejemplo: Lista con balas
% ============
\begin{compactitem}
	\item peras
	\item manzanas
	\item naranjas
\end{compactitem}


\noindent Ejemplo de lista en varias columnas.
% Ejemplo: Listas en varias columnas
% ============
\begin{multicols}{2} % El parámetro es el número de columnas de la lista
	\begin{compactenum}
		\item peras
		\item manzanas
		\item naranjas
		\item patatas
		\item calabazas
		\item fresas
	\end{compactenum}
\end{multicols}


\newpage


\section{Ejemplos de tablas}
\label{sec:ejTablas}
\index{ejemplos!tablas}
A continuación se incluyen algunos ejemplos de tablas hechas con \LaTeX{} y paquetes dedicados.

% Ejemplo: Tabla con macro \cline
% ==========
\begin{table}[htb]%
	\centering
	\caption{Ejemplo de uso de la macro \texttt{cline}}
	\label{tab:cline}
	\begin{tabular}[t]{|r|l|}
		\hline
		7C0 & hexadecimal \\[1cm] % Ejemplo de separación fijada entre líneas
		3700 & octal \\ \cline{2-2}
		11111000000 & binario \\
		\hline \hline
		1984 & decimal \\
		\hline
	\end{tabular}
\end{table}


\noindent Ejemplo de tabla en la que se controla el ancho de la celda.

% Ejemplo: Ejemplo de tabla con control de la anchura de celda.
% ==========
\begin{table}[htb]%
	\centering
	\caption{Ejemplo de tabla con especificación de anchura de columna}
	\label{tab:anchura}
	\begin{tabular}{ | l | l | l | p{5cm} |}
		\hline
		Día & Temp Mín (\textdegree C) & Temp Máx (\textdegree C) & Previsión \\ \hline
		Lunes & 11 & 22 & Día claro y muy soleado. Sin embargo, la brisa de la tarde puede hacer que las temperaturas desciendan \\ \hline
		Martes & 9 & 19 & Nuboso con chubascos en muchas regiones. En Cataluña claro con posibilidad de bancos nubosos al norte de la región \\ \hline
		Miércoles & 10 & 21 & La lluvía continuará por la mañana pero las condiciones climáticas mejorarán considerablemente por la tarde\\
		\hline
	\end{tabular}
\end{table}

\clearpage


\section{Ejemplos de figuras}
\label{sec:ejFiguras}
\index{ejemplos!figuras}

En esta sección se añaden ejemplos de muestra para la inclusión de figuras simples y subfiguras.

% Ejemplo: Ejemplo de inclusión de figura
% ============
\begin{figure}[htb]
	\centering
	\includegraphics[width=0.4\linewidth]{escudoInf}
	\caption[Ejemplo de figura]{Figura vectorial del escudo de la ESI}
	\label{fig:ejFigure}
\end{figure}


\noindent Ejemplo de figuras compuestas por subfiguras.

% Ejemplo: Ejemplo de inclusión de subfiguras
% ============
\begin{figure}[htb]
	\centering
	\subfigure[Gráfico vectorial PDF]{
		\includegraphics[width=0.4\linewidth]{escudoInf}
		\label{fig:escudoColor}
	} 
	\subfigure[Gráfico png]{
		\includegraphics[width=0.4\linewidth]{escudoInfBW}
		\label{fig:escudoBW}
	}
	\caption[Ejemplo de subfiguras]{Ejemplo de inclusión de subfiguras en un mismo entorno}
	\label{fig:ejSubfigures}
\end{figure}


\clearpage


\section{Ejemplos de listados}
\label{sec:ejListados}
\index{ejemplos!listados}

Ejemplos más representativos de inclusión de porciones de código fuente.

% Ejemplo: Listado Java
% ============
\begin{lstlisting}[language=Java,float=ht,caption={[Código fuente en Java]Ejemplo de código fuente en lenguaje Java},label=lst:java]
// @author www.javadb.com
public class Main {    
// Este método convierte un String a
// un vector de bytes

public void convertStringToByteArray() {

String stringToConvert = "This String is 15";      
byte[] theByteArray = stringToConvert.getBytes();        
System.out.println(theByteArray.length);        
}

// argumentos de línea de comandos 
public static void main(String[] args) {
new Main().convertStringToByteArray();
}
}
\end{lstlisting}



\noindent Otro ejemplo.

\begin{lstlisting}[style=C,float=ht,caption={Ejemplo de código C},label=lst:codC]
// Este código se ha incluido tal cual está 
// en el fichero \LaTeX{}
#include <stdio.h>
int main(int argc, char* argv[]) {
puts("¡Hola mundo!");
}
\end{lstlisting}


\noindent Ejemplo de entrada por consola.

\begin{lstlisting}[style=consola, numbers=none]
$ gcc -o Hola HolaMundo.c
\end{lstlisting}