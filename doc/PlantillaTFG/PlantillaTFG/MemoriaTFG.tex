% !TeX encoding = utf8
% !TeX TXS-program:bibliography = txs:///biber
%%%%%%%%%%%%%%
% Fichero: MemoriaTFG
% Autor: Jesús Salido Tercero (http://www.uclm.es/profesorado/jsalido)
% Fecha (creación): Febrero 2010 
% Rev. : Febrero 2017
% Descripción: Plantilla para memoria de TFG 
% (Escuela Sup. de Informática, UCLM). Creada para el curso 
% “LaTeX esencial para preparación de TFG, Tesis y otros documentos 
% académicos” (Esc. Sup. Informática-UCLM)
%
% Comentarios: Preparada para `pdflatex' y `biblatex' (con `biber'). 
% Documento editado con TeXstudio. 
% Para su compilación se aconseja utilizar como compilador 
% por defecto `latexmk.'
%%%%%%%%%%%%%%

% EDITAR: Idioma pral. \spanishtrue (Español), \spanishfalse (Inglés)
\newif\ifspanish
\spanishtrue
%\spanishfalse

\documentclass[11pt,a4paper,twoside,openright,final]{book}
\usepackage[utf8]{inputenx} % Codificación de entrada
\usepackage[spanish,english]{babel} % Internacionalización
\usepackage{indentfirst} % Para asegurar sangrado en 1ª línea tras sección (necesario con varios idiomas)
\usepackage[top=2.5cm,bottom=2.5cm,inner=3.5cm,outer=2cm]{geometry}


% Tipografía
\usepackage{libertine}
\usepackage[libertine]{newtxmath}

% Inclusión de símbolos
\usepackage{textcomp,marvosym,pifont} % Símbolos
\usepackage{cclicenses} % Para símbolos de licencias CC
\usepackage{amsmath,amsfonts,amssymb} % Caracteres matemáticos
\usepackage[T1]{fontenc} % Codificación de salida 
\usepackage{microtype} % Mejoras tipográficas para pdflatex

\usepackage{url} % Para escritura de URL
\urlstyle{sf}
\usepackage[bookmarks,bookmarksnumbered=true,hyperfootnotes=false,hidelinks]{hyperref}

% Definición de colores
% OJO: Este paquete debe cargarse antes de ctable. 
\usepackage[usenames,dvipsnames,svgnames,x11names,table]{xcolor}
\definecolor{sombra}{gray}{.70}


% Tablas y gráficos
\usepackage{framed} % Recuadro en torno a texto 
\usepackage{fancybox} % Recuadro exóticos
\usepackage{graphicx}  % Inclusión de figuras
\usepackage{subfigure} % Inclusión de subfiguras
\graphicspath{{./figs/}} % Path de búsqueda de ficheros gráficos
\DeclareGraphicsExtensions{.pdf,.png,.jpg} % Precedencia de extensiones
\usepackage{rotating}  % Giro de cajas (texto, figuras, tablas) (No DVI)
\usepackage{tikz}
\usepackage{ctable} % Inclusión de tablas. (ctable incluye): color,xkeyval,array,tabularx,booktabs,rotating
\usepackage{multirow}


% Paquetes especiales para Informática
\usepackage{listings} % Inclusión de listados de código
% Algoritmos
\ifspanish
\usepackage[lined,boxruled,algochapter,commentsnumbered,spanish]{algorithm2e}
\else
\usepackage[lined,boxruled,algochapter,commentsnumbered,english]{algorithm2e}
\fi
% Paquete específico para la generación de algorithmos en LaTeX. Por su complejidad se requiere consultar la documentación del paquete para su correcto uso.


% Ajustes de formato
\usepackage{paralist, multicol} % Mayor control de listas
\usepackage{titlesec} % Personalización completa de títulos de secciones
\usepackage{sectsty} % Personalización de títulos de sección con una interfaz más simple que la suministrada por titlesec
% Tipo de letra empleado en títulos de secciones (paquete sectsty)
\usepackage[margin=10pt,font=small,labelfont=bf,format=hang]{caption} % Personalización de títulos de figuras y tablas
\usepackage{fancyhdr} % Cabeceras y pies personalizados


% Bibliografía (Multilingüe en español e inglés, empleando el idioma de la fuente)
\usepackage[backend=biber,sortcites,autolang=other,language=auto]{biblatex}
% Línea añadida para eliminar el idioma de la fuente bibliográfica.
\AtEveryBibitem{\clearfield{note} \clearlist{language}}
\addbibresource{biblioTFG.bib}
\usepackage[autostyle]{csquotes}

 % Paquete para índice alfabético
\usepackage{makeidx}
\makeindex

% Comandos y ajustes de usuario
% Ajustes definidos por el usuario

% Control de espaciado entre párrafos e indentación
\setlength{\parskip}{1.3ex plus 0.2ex minus 0.2ex} % Espacio entre párrafos (más versatil cuando se fija como una rubber length (longitud elástica)
%\setlength{\parskip}{2mm plus 0.2mm minus 0.2mm} (Clase esi-tfg.cls por D. Villa)
%\setlength{\parindent}{8ex} % Valor por defecto de la indentación en español. 
%Sólo es preciso activar para modificar el valor por defecto (indicar valor cero para anular)


% Ajuste de cabecera y pie de página (paquete fancyhdr)
\fancyhf{} % Reset de la cabecera y pie
% En las páginas impares, parte izquierda del encabezado, aparecerá el nombre de capítulo
\fancyhead[LO]{\sffamily\leftmark} 
% En las páginas pares, parte derecha del encabezado, aparecerá el nombre de sección
\fancyhead[RE]{\sffamily\rightmark} 
% Números de página en las esquinas de los encabezados
\fancyhead[RO,LE]{\sffamily\thepage}

% Formato para el capítulo: N. Nombre
\renewcommand{\chaptermark}[1]{\markboth{\textbf{\thechapter.#1}}{}}
% Formato para la sección: N.M. Nombre
\renewcommand{\sectionmark}[1]{\markright{\textbf{\thesection. #1}}} 

% Ancho de la línea horizontal bajo el encabezado
\renewcommand{\headrulewidth}{0.6pt} 
% Ancho de la línea horizontal sobre el pie (en este ejemplo está vacío)
%\renewcommand{\footrulewidth}{0.6pt} 

%\setlength{\headheight}{16pt} % Por defecto \headheight 12pt, pero se agranda al emplear fancyhdr
\setlength{\headheight}{1.5\headheight} % Aumenta la altura del encabezado en una vez y media


% Evita que la última página de cap. tenga cabecera y pie 
% si dicha página está en blanco (para clase book)
\makeatletter
\def\cleardoublepage{\clearpage\if@twoside
\ifodd\c@page
\else\hbox{}\thispagestyle{empty}\newpage
\if@twocolumn\hbox{}\newpage\fi\fi\fi}
\makeatother



%====================================
% Configuración inicio capítulos (paquete titlesec)
\newcommand{\bigrule}{\titlerule[0.5mm]}

\titleformat{\chapter}[display] % cambiamos el formato de los capítulos
{\bfseries\Huge\sffamily} % por defecto se usarán caracteres de tamaño \Huge en negrita y familia sans serif
{% contenido de la etiqueta
\titlerule % línea horizontal
\filleft % texto alineado a la derecha
\Large{\MakeUppercase\chaptertitlename}\ % "Capítulo" o "Apéndice" en tamaño \Large en lugar de \Huge
\Large\thechapter} % número de capítulo en tamaño \Large
{0mm} % espacio mínimo entre etiqueta y cuerpo
{\filleft\MakeUppercase} % texto del cuerpo alineado a la derecha
[\vspace{0.5mm} \bigrule] % después del cuerpo, dejar espacio vertical y trazar línea horizontal gruesa

%====================================
% Tipo de letra empleado en títulos de secciones (paquete sectsty)
\sectionfont{\sffamily\bfseries\MakeUppercase}
\subsectionfont{\sffamily\bfseries}
\subsubsectionfont{\sffamily\bfseries}

%====================================
% Alguna definiciones de nivel de gris
\definecolor{gris97}{gray}{.97}
\definecolor{gris95}{gray}{.95}
\definecolor{gris75}{gray}{.75}
\definecolor{gris45}{gray}{.45}


% Personalización del entorno lstlisting (ver documentación paquete listings para más infor.)
   
\lstset{ % Estilo por defecto
	basicstyle={\footnotesize\ttfamily}, % Estilo básico para el texto
	%stringstyle=\textsl,        % Estilo para las cadenas
	stringstyle={\color{Red1}\ttfamily\bfseries},
	commentstyle={\color{Green4}\sffamily\bfseries},% Estilo para los comentarios
	keywordstyle={\color{Blue1}\bfseries},% Estilo para las palabras clave
	%	keywordstyle=[1]\textbf,    % Posibilidad de particularizar el estilo 
	%	keywordstyle=[2]\textbf,    %
	%	keywordstyle=[3]\textbf,    %
	%	keywordstyle=[4]\textbf,    %
	%	deletekeywords={}, 			% Quita keywords separadas por comas
	captionpos=t,               % Ajusta la posición de títulos 
	numbers=left,               % Posición de números de línea
	numberstyle={\tiny\sffamily\bfseries},          % Tamaño del número de línea
	numberfirstline=false,
	firstnumber=1, 				%  Nº de la primera línea
	stepnumber=1,               % Paso de línea numerada
	numbersep=10pt,             % Separación del texto al número de línea
	tabsize=2,                  % Tamaño del tabulador
	extendedchars=\true,        % Gestiona en empleo de caracteres extendidos utf8
	texcl=true,				    % Necesario para unicode en los comentarios
	breaklines=true,            % Ajusta división automática de líneas
	breakatwhitespace=true,     %
	frame=single,               % none, leftline, topline, bottomline, lines, single, shadowbox 
	frameround=tttt, 			% Redondea los bordes del frame
	rulecolor={\color{IndianRed}},    % Color del frame
	showspaces=false,           % Muestra espacios en blanco
	showtabs=false,             % Muestra tabuladores
	showstringspaces=true,      % Muestra espacios en blanco en las cadenas        
	xleftmargin=1cm,xrightmargin=1cm,
	breaklines=true,
	%	framexleftmargin=17pt
	%	framexrightmargin=5pt,
	%	framexbottommargin=4pt,
	backgroundcolor={\color{Azure1}} % Color del fondo
}


%Estilo definido para lenguaje C
\lstdefinestyle{C}{%
	language=C,
	frame=L,
	rulesep=.1pt,
	rulecolor=\color{black},
}


% Estilo definido para comandos de consola
\lstdefinestyle{consola}{%
	basicstyle={\color{White}\scriptsize\bf\ttfamily},
	backgroundcolor={\color{Black}},
	frame=none,
	showspaces=true
}




% Comandos definidos por el usuario
%====================================
% Comandos que definen datos del documento
\makeatletter
\newcommand{\tituloPrimera}[1]{\newcommand{\@tituloPrimera}{#1}}
\newcommand{\tituloSegunda}[1]{\newcommand{\@tituloSegunda}{#1}}
\newcommand{\titulo}[1]{\newcommand{\@titulo}{#1}\renewcommand{\@title}{#1}}
\newcommand{\tipoDoc}[1]{\newcommand{\@tipoDoc}{#1}}
\newcommand{\autor}[1]{\newcommand{\@autor}{#1}\renewcommand{\@author}{#1}}
\newcommand{\email}[1]{\newcommand{\@email}{\url{#1}}}
\newcommand{\director}[1]{\newcommand{\@director}{#1}}
\newcommand{\codirector}[1]{\newcommand{\@codirector}{#1}}
\newcommand{\tutor}[1]{\newcommand{\@tutor}{#1}}
\newcommand{\instEdu}[1]{\newcommand{\@instEdu}{#1}}
\newcommand{\centroEdu}[1]{\newcommand{\@centroEdu}{#1}}
\newcommand{\deptoEduPrimera}[1]{\newcommand{\@deptoEduPrimera}{#1}}
\newcommand{\deptoEduSegunda}[1]{\newcommand{\@deptoEduSegunda}{#1}}
\newcommand{\escudo}[1]{\newcommand{\@escudo}{#1}}
\newcommand{\titulacion}[1]{\newcommand{\@titulacion}{#1}}
\newcommand{\especialidad}[1]{\newcommand{\@especialidad}{#1}}
\newcommand{\fechaDef}[1]{\newcommand{\@fechaDef}{#1}}
\newcommand{\mesDef}[1]{\newcommand{\@mesDef}{#1}}
\newcommand{\yearDef}[1]{\newcommand{\@yearDef}{#1}}
\newcommand{\lugarDef}[1]{\newcommand{\@lugarDef}{#1}}
\makeatother


%====================================
% Portada (tfg)
% Portada (aquí gralmente no habrá que editar nada)
\makeatletter
\newcommand{\portadaTFG}{%
	\begin{titlepage}
		\begin{center}
			\includegraphics[width=3.5cm]{\@escudo}\vspace{1cm}
			
			{\LARGE \textbf{\@instEdu\\[1.5\parskip]
					\@centroEdu\\[2cm]
					\@titulacion}}\\[0.5cm]
			{\large \textbf{\@especialidad}}\\[1.5cm]
			{\LARGE \textbf{\@tipoDoc}}\\[1cm]
			
			{\LARGE \@tituloPrimera}\\ \smallskip%			
			\ifdefined\@tituloSegunda{\LARGE	\@tituloSegunda}\\[3cm]
			\else \vspace{3cm}
			\fi
			
			{\Large \@autor}\vfill%
		\end{center}
		
		\begin{flushright}
			{\Large \@fechaDef}
		\end{flushright}
		
		\cleardoublepage
\end{titlepage}}
\makeatother

%====================================
% Página inicial (tfg)
% Página inicial (es como la portada, añade Director/es o Tutor)
\makeatletter
\newcommand{\portadillaTFG}{%
	\begin{center}
		\includegraphics[width=3.5cm]{\@escudo}\\[1.5cm]
		
		{\LARGE \textbf{\@instEdu \\[1.5\parskip]
				\@centroEdu}}\\[0.5cm]
		{\Large \textbf{\@deptoEduPrimera}}\\ \smallskip%
		\ifdefined\@deptoEduSegunda{\Large \textbf{\@deptoEduSegunda}}\\[0.5cm]
		\else \vspace{0.5cm}
		\fi
		{\large \textbf{\@especialidad}}\\[1.5cm]
		
		{\LARGE \textbf{\@tipoDoc}}\\[1cm]
		
		
		{\LARGE \textbf{\@tituloPrimera}}\\ \smallskip%		
		\ifdefined\@tituloSegunda{\LARGE \textbf{\@tituloSegunda}}\\[3cm]
		\else \vspace{3cm}
		\fi
	\end{center}
	
	\begin{flushleft}
		{\Large Autor(a): \@autor} \\ \bigskip%
		{\Large Director(a): \@director} \\ \bigskip%
		% Si hay definido un codirector se añade automáticamente la línea siguiente
		\ifdefined\@codirector {\Large Director(a): \@codirector} \fi 
	\end{flushleft}
	\vfill%
	
	\begin{flushright}
		{\Large \@fechaDef}
	\end{flushright}
	
	\newpage}
\makeatother

%====================================
% Página tribunal (tfg)
% Tribunal
\makeatletter
\newcommand{\tribunalTFG}{
	{\flushright \LARGE \textsc{Tribunal:}}
	
	\vspace*{\stretch{0.5}}
	\hspace*{1cm}{\Large Presidente: \hrulefill}
	
	\vspace*{\stretch{0.5}}
	\hspace*{1cm}{\Large Vocal: \hrulefill}
	
	\vspace*{\stretch{0.5}}
	\hspace*{1cm}{\Large Secretario: \hrulefill}
	
	\vspace*{\stretch{0.5}}
	{\flushright \LARGE \textsc{Fecha de defensa:} \hrulefill}
	
	\vspace*{\stretch{1.5}}
	{\flushright \LARGE \textsc{Calificación:} \hrulefill}
	
	\vspace*{\stretch{2.5}}
	\begin{center}
		\begin{tabularx}{\linewidth}{X X X}
			{\large \textsc{Presidente}} & {\large \textsc{Vocal}} & {\large \textsc{Secretario}}\\[2.5cm]
			Fdo.: & Fdo.: & Fdo.:		
		\end{tabularx}
	\end{center}
	\cleardoublepage}
\makeatother



%====================================
% Comando para incluir la dedicatoria
% Con la opción stretch se puede colocar verticalmente la dedicatoria de forma relativa para que quede el doble de espacio por debajo que por encima
\newcommand{\dedicado}[1]{ % Dedicatoria
	\null\vspace{\stretch{1}}
	\begin{flushright}
	\emph{#1}
	\end{flushright}
	\vspace{\stretch{2}}\null
	\cleardoublepage
}



%====================================
% Créditos y licencia (opcional, 1 pág.)
% Todas las obras en gral. deberían presentar información relativa a la propiedad intelectual del contenido y condiciones bajo las cuales se puede distribuir y reproducir
\makeatletter
\newcommand{\creditos}[2]{%
	\null\vspace{6cm}
	{\small \noindent \@titulo\\
	\textcopyright{} \@autor, \@yearDef\\[1cm]
	#1}\\
	\includegraphics[width=0.15\linewidth]{#2}
	
	\clearpage
}
\makeatother




%====================================
% Añade el comando \tecla para crear indicaciones de pulsación de teclas
\usetikzlibrary{shadows} % Necesario para poder crear nuevo comando de indicación de pulsación de tecla.
\newcommand*\tecla[1]{%   
  \tikz[baseline=(key.base)]
    \node[%
      draw,
      fill=white,
      drop shadow={%
	      shadow xshift=0.25ex,
	      shadow yshift=-0.25ex,
	      fill=black,
	      opacity=0.75
      },
      rectangle,
      rounded corners=2pt,
      inner sep=1pt,
      line width=0.5pt,
      font=\scriptsize\sffamily
    ](key) {#1\strut}
  ;
}

%====================================
% Desactivación de división de palabras. 
% Uso: \nodivide o \nodivide[<n>]
\newcommand{\nodivide}[1][10000]{%
	\hyphenpenalty=#1 % Valor típico: hasta 10000
	\exhyphenpenalty=#1 % Valor típico: hasta 10000
	\sloppy
}

%====================================
% Desactivación de división de palabras. 
% Uso: \nowidowandorphan o \nowidowandorphan[<n>]
\newcommand{\nowidowandorphan}[1][10000]{%
	\clubpenalty=#1  % % Valor típico: hasta 10000
	\widowpenalty=#1 % % Valor típico: hasta 10000
}

%====================================
% Código para evitar la división de notas al pie en págs. diferentes
% Uso: \nodividenotas o \nodividenotas[<n>]
\newcommand{\nodividenotes}[1][10000]{%
	\interfootnotelinepenalty=#1 % Valor típico: hasta 10000
}


%====================================
% Creación de un contador nuevo para almacenar el nº de páginas actual
% OJO: Debe ir antes de \mainmatter (antes de que se reinicie en cnt page)
\newcommand{\savepagecnt}{%
	\newcounter{totpages}
	\setcounter{totpages}{\value{page}}
	\addtocounter{totpages}{1}
}

%====================================
% Continuación de la paginación desde el valor almacenado en \totpages
\newcommand{\contpagination}{%
	\setcounter{page}{\value{totpages}}
}

%====================================
% Limpia las cabeceras de la primera página de capítulo
\newcommand{\cleanhdfirst}{%
	\fancypagestyle{plain}{%
		\fancyhf{}%
		\renewcommand{\headrulewidth}{0pt}
		\renewcommand{\footrulewidth}{0pt}
	}
}

%====================================
% Añadido de entorno abstract en clase book
% donde no está definido por defecto
\newenvironment{abstract}%
{\cleardoublepage\null \vfill\begin{center}%
\bfseries \sffamily \abstractname \end{center}}%
{\vfill \null}


%====================================
% Comando para incluir una nota de aviso al margen.
% Si el margen no es suficientemente amplio puede generar un Overfull \hbox
\newcommand{\ojo}[1]{%
	\marginpar{\footnotesize\raggedright\ding{42} #1}}
\newcommand{\uju}{%
	\renewcommand{\ojo}[1]{}}











%%%%%%%%%%%%%%
% Datos del documento
% Estos valores son los que se emplean en el documento por lo que no son
% traducidos. Cuando algún campo puede tener varias líneas aparecen dos
% campos señalados como <campo>Primera y <campo>Segunda.
% Si no se desea emplear un campo se comenta.
%====================================
% EDITAR: Datos del documento
\tituloPrimera{<Primera línea título>}% 1ª Línea
\tituloSegunda{<Segunda línea título>}% 2ª Línea  % Comentar si no existe
\titulo{<Título corto>}% Título corto
\autor{<autor (nombre apellidos)>}
\email{<autor>@uclm.es}
\director{<director (nombre apellidos)>}
\codirector{<codirector (nombre apellidos)>} % Comentar si no existe
\instEdu{UNIVERSIDAD DE CASTILLA-LA MANCHA}
\escudo{escudoInf} % Inclusión del escudo de la institución
%\escudo{escudoUCLM} 
\centroEdu{ESCUELA SUPERIOR DE INFORMÁTICA}
\deptoEduPrimera{<Primera línea Depto. Director>}% 1ª Línea
\deptoEduSegunda{<Segunda línea Depto. Director>}% 2ª Línea  % Comentar si no existe
\titulacion{GRADO EN INGENIERÍA INFORMÁTICA}
\especialidad{<Tecnología Específica>} % Tecnología específica, ...
\tipoDoc{TRABAJO FIN DE GRADO} % Proyecto Fin de Carrera, Trabajo Fin de Grado, Tesis Doctoral, ...
% Si las fechas se desean en inglés hay que ponerlas explícitamente.
\fechaDef{<mes, año>} % Fecha de defensa
\mesDef{<mes>}        % Mes de defensa
\yearDef{<año>}        % Año de defensa
\lugarDef{<Ciudad>}% Lugar de defensa
%%%%%%%%%%%%%%
\begin{document}
% --- Bloque inicial del documento ---
\ifspanish
\selectlanguage{spanish}% Emplea idioma español
\else
\selectlanguage{english}% Emplea idioma inglés
\fi 
\frontmatter
% Cambia la numeración de páginas a números romanos y las secciones no están numeradas aunque si aparecen en el índice de contenidos.
\pagestyle{empty}  % Páginas sin cabecera ni pies

%%%%%%%%%%%%%%
\portadaTFG

\portadillaTFG
%%%%%%%%%%%%%%

% Créditos
% EDITAR: Licencia (si se desea modificar).
\creditos{Este documento se distribuye con licencia Creative Commons Atribución Compartir Igual 4.0. El texto completo de la licencia puede obtenerse en \url{https://creativecommons.org/licenses/by-sa/4.0/}. La copia y distribución de esta obra está permitida en todo el mundo, sin regalías y por cualquier medio, siempre que esta nota sea preservada. Se concede permiso para copiar y distribuir traducciones de este libro desde el español original a otro idioma, siempre que la traducción sea aprobada por el autor del libro y tanto el aviso de copyright como esta nota de permiso, sean preservados en todas las copias.}{cclicense}

%%%%%%%%%%%%%%
\tribunalTFG


%%%%%%%%%%%%%%
% Dedicatoria (opcional, 1 pág.)
% Aunque opcional, no se debería perder la oportunidad de poder dedicar el trabajo a alguien MUY especial.
% EDITAR: Dedicatoria
\dedicado{Alguien MUY especial ... \\
(para siempre)} % Editar para dedicar a alguien

% Ajustes para texto normal
% Estos tres comandos admiten un parámetro entre [corchetes] que se puede ir elevando desde 5000 (defecto) hasta 10000. Cuanto mayor es el valor más forzado está LaTeX
%\nodivide[10000]      % Evita la división de palabras 
%\nowidowandorphan % Evita viudas y huérfanas
%\nodividenotes %Evita división entre páginas de las notas
\pagestyle{plain} % Páginas sólo con numeración inferior en pie

%%%%%%%%%%%%%%
% Resumen (Si se desea cambiar el orden en que se imprimen, cambiar de forma manual)
\selectlanguage{spanish}
\begin{abstract}
	% EDITAR: Resumen
	(... versión del resumen en español ...)
El resumen debe ocupar como máximo una página y en dicho espacio proporcionará información crucial sobre el \emph{`qué'} (problemática que trata de resolver el TFG), el \emph{`cómo'} (metodología para llegar a los resultados) y los objetivos alcanzados.
\end{abstract}

% Abstract
\selectlanguage{english}
\begin{abstract}
	% EDITAR: Abstract
	(... english version of the abstract ...)
\end{abstract}


% Selección de idioma para el resto del documento
\ifspanish
\selectlanguage{spanish}% Para el resto del documento el idioma es español.
\else
\selectlanguage{english}
\fi

% Agradecimientos (1 pág.)
\include{./caps/Agradecimientos} % Agradecimientos etc.




%%%%%%%%%%%%%%
% Índices
\pagestyle{fancy} % Estilo de página ajustado por fancyhdr
% OJO: Si se cambian los nombres que otorga babel para el idioma en uso, los nombres vuelven a su estado cada vez que se vuelva a seleccionar el idioma. Por eso es conveniente que el cambio de nombre aparezca justo antes de su empleo.

% Índice General
\tableofcontents  % Índice general

% Índice de figuras
%\addcontentsline{toc}{chapter}{\listfigurename} % Añade la lista de figuras al TOC.
\listoffigures    % Índice de figuras (opcional)

% Índice de tablas
\ifspanish
\renewcommand{\tablename}{Tabla} % Se sustituye 'Cuadro' por 'Tabla'
\renewcommand{\listtablename}{Índice de tablas}
\fi
%\addcontentsline{toc}{chapter}{\listtablename} % Añade la lista de tablas al TOC.
\listoftables % Índice de tablas (opcional)

% Índice de listados
% Personalización de los títulos de los listados y del índice de listados
\ifspanish
\renewcommand{\lstlistingname}{Listado} % Nombre que figura en el pie
\renewcommand{\lstlistlistingname}{Índice de listados} % Nombre asignado
\fi
%\addcontentsline{toc}{chapter}{\lstlistlistingname} % Añade la lista de listados al TOC.
\lstlistoflistings % Índice de listados creados con listings (opcional)

% Índice de algoritmos
% Personalización de los títulos de los algoritmos y del índice de algoritmos
\ifspanish
\SetAlgorithmName{Algoritmo}{Alg}{Índice de algoritmos}
\fi
%\addcontentsline{toc}{chapter}{Índice de algoritmos} % Añade la lista de algoritmos al TOC.
\listofalgorithms % Índice de algoritmos creados con algortihm2e




%%%%%%%%%%%%%%
% OPCIONAL: No es preciso incluirlo
% Lista de acrónimos o nomenclatura (Opcional)
% En libros o tesis es usual encontrarlos. En el TFG es inusual.
% Debe estar ordenado por orden alfabético.
% Me parece más interesante poner el índice alfabético final que es mucho más útil.
%\include{Nomeclatura} % (Opcional) Acrónimos y nomenclatura usada

% Prólogo (opcional)
% En libros o tesis es más usual encontrarlos. En el TFG es inusual.
%\include{Prologo} % Prólogo (opcional).




%%%%%%%%%%%%%%
% Capítulos del documento
% Creación de un contador nuevo para almacenar el nº de páginas actual
% OJO: Debe ir antes de \mainmatter (antes de que se reinicie el cnt page)
\savepagecnt
\mainmatter
% Justo antes del primer capítulo del libro. Activa la numeración con números arábigos y reinicia el contador de páginas.

% Se incluye un fichero para cada capítulo. Se emplea la instrucción \include porque en los libros lo más habitual es que cada cápítulo comience en una nueva página.

% OJO: No cambiar de ubicación. Redefinición de estilo plain para que la 1ª página del capítulo no tenga ni cabecera ni pie
\cleanhdfirst

% Reajuste del número de página consecutivo para no reiniciar paginación en Cáp. 1
%\contpagination % Comentar si se desea reiniciar paginación

% Con el comando \include se garantiza que el capítulo comienza en nueva página.

% Aquí irían todos los capítulos del documento
\chapter{Introducción}
\label{cap:Introduccion}

Este capítulo aborda la motivación del trabajo. Se trata de señalar la necesidad de la que surge, su actualidad y pertinencia. Puede incluir también un estado de la cuestión en la que se revisen estudios o desarrollos previos y en qué medida sirven de base al trabajo que se presenta.

A continuación se muestran algunos ejemplos para la inclusión de elementos en el documento.

% ------------------------------------------------------------------------------
% Ejemplos para la plantilla
% ------------------------------------------------------------------------------
\section{Ejemplos de listas}
\label{sec:ejListas}
\index{ejemplos} % Véase cómo se incluyen entradas en el índice alfabético
A continuación se van a añadir algunos ejemplos que pueden emplearse al redactar la memoria.

\index{ejemplos!listas} % Para el índice
\noindent Ejemplo de lista con \emph{bullet} especial.
% Ejemplo: Lista con bullets especiales
% ============
\begin{itemize}
	\item[*] peras
	\item manzanas
	\item[\ding{170}] naranjas
\end{itemize}

\noindent Ejemplo de lista compacta (también se puede emplear el entorno para enumeraciones \emph{compactenum})
% Ejemplo: Lista con balas
% ============
\begin{compactitem}
	\item peras
	\item manzanas
	\item naranjas
\end{compactitem}


\noindent Ejemplo de lista en varias columnas.
% Ejemplo: Listas en varias columnas
% ============
\begin{multicols}{2} % El parámetro es el número de columnas de la lista
	\begin{compactenum}
		\item peras
		\item manzanas
		\item naranjas
		\item patatas
		\item calabazas
		\item fresas
	\end{compactenum}
\end{multicols}


\newpage


\section{Ejemplos de tablas}
\label{sec:ejTablas}
\index{ejemplos!tablas}
A continuación se incluyen algunos ejemplos de tablas hechas con \LaTeX{} y paquetes dedicados.

% Ejemplo: Tabla con macro \cline
% ==========
\begin{table}[htb]%
	\centering
	\caption{Ejemplo de uso de la macro \texttt{cline}}
	\label{tab:cline}
	\begin{tabular}[t]{|r|l|}
		\hline
		7C0 & hexadecimal \\[1cm] % Ejemplo de separación fijada entre líneas
		3700 & octal \\ \cline{2-2}
		11111000000 & binario \\
		\hline \hline
		1984 & decimal \\
		\hline
	\end{tabular}
\end{table}


\noindent Ejemplo de tabla en la que se controla el ancho de la celda.

% Ejemplo: Ejemplo de tabla con control de la anchura de celda.
% ==========
\begin{table}[htb]%
	\centering
	\caption{Ejemplo de tabla con especificación de anchura de columna}
	\label{tab:anchura}
	\begin{tabular}{ | l | l | l | p{5cm} |}
		\hline
		Día & Temp Mín (\textdegree C) & Temp Máx (\textdegree C) & Previsión \\ \hline
		Lunes & 11 & 22 & Día claro y muy soleado. Sin embargo, la brisa de la tarde puede hacer que las temperaturas desciendan \\ \hline
		Martes & 9 & 19 & Nuboso con chubascos en muchas regiones. En Cataluña claro con posibilidad de bancos nubosos al norte de la región \\ \hline
		Miércoles & 10 & 21 & La lluvía continuará por la mañana pero las condiciones climáticas mejorarán considerablemente por la tarde\\
		\hline
	\end{tabular}
\end{table}

\clearpage


\section{Ejemplos de figuras}
\label{sec:ejFiguras}
\index{ejemplos!figuras}

En esta sección se añaden ejemplos de muestra para la inclusión de figuras simples y subfiguras.

% Ejemplo: Ejemplo de inclusión de figura
% ============
\begin{figure}[htb]
	\centering
	\includegraphics[width=0.4\linewidth]{escudoInf}
	\caption[Ejemplo de figura]{Figura vectorial del escudo de la ESI}
	\label{fig:ejFigure}
\end{figure}


\noindent Ejemplo de figuras compuestas por subfiguras.

% Ejemplo: Ejemplo de inclusión de subfiguras
% ============
\begin{figure}[htb]
	\centering
	\subfigure[Gráfico vectorial PDF]{
		\includegraphics[width=0.4\linewidth]{escudoInf}
		\label{fig:escudoColor}
	} 
	\subfigure[Gráfico png]{
		\includegraphics[width=0.4\linewidth]{escudoInfBW}
		\label{fig:escudoBW}
	}
	\caption[Ejemplo de subfiguras]{Ejemplo de inclusión de subfiguras en un mismo entorno}
	\label{fig:ejSubfigures}
\end{figure}


\clearpage


\section{Ejemplos de listados}
\label{sec:ejListados}
\index{ejemplos!listados}

Ejemplos más representativos de inclusión de porciones de código fuente.

% Ejemplo: Listado Java
% ============
\begin{lstlisting}[language=Java,float=ht,caption={[Código fuente en Java]Ejemplo de código fuente en lenguaje Java},label=lst:java]
// @author www.javadb.com
public class Main {    
// Este método convierte un String a
// un vector de bytes

public void convertStringToByteArray() {

String stringToConvert = "This String is 15";      
byte[] theByteArray = stringToConvert.getBytes();        
System.out.println(theByteArray.length);        
}

// argumentos de línea de comandos 
public static void main(String[] args) {
new Main().convertStringToByteArray();
}
}
\end{lstlisting}



\noindent Otro ejemplo.

\begin{lstlisting}[style=C-ruled,float=ht,caption={Ejemplo de código C},label=lst:codC]
// Este código se ha incluido tal cual está 
// en el fichero \LaTeX{}
#include <stdio.h>
int main(int argc, char* argv[]) {
puts("¡Hola mundo!");
}
\end{lstlisting}


\noindent Ejemplo de entrada por consola.

\begin{lstlisting}[style=consola, numbers=none]
$ gcc -o Hola HolaMundo.c
\end{lstlisting}


\subsection{Algoritmos con el paquete \texttt{algorithm2e}}
Como ya se ha comentado en los textos científicos relacionados con las TIC\footnote{Por supuesto en un TFG o tesis de una Escuela de Informática.} (Tecnologías de la Información y Comunicaciones) suelen aparecer porciones de código en los que se explica alguna función o característica relevante del trabajo que se expone. Muchas veces lo que se quiere ilustrar es un algoritmo o método en que se ha resuelto un problema abstrayéndose del lenguaje de programación concreto en que se realiza la implementación. El paquete \texttt{algorithm2e}\footnote{\url{https://osl.ugr.es/CTAN/macros/latex/contrib/algorithm2e/doc/algorithm2e.pdf}} proporciona un entorno \texttt{algorithm} para la impresión apropiada de algoritmos tratándolos como objetos flotantes y con muchas flexibilidad de personalización. En el algoritmo \ref{alg:como} se muestra cómo puede emplearse dicho paquete. En este curso no se explican las posibilidades del paquete más en profundidad ya que excede el propósito del curso. A todos los interesados se les remite a la documentación del mismo.


% Ejemplo:
% ============
\IncMargin{1em}
\begin{algorithm}
\SetKwInOut{Input}{Datos}\SetKwInOut{Output}{Resultado}
\LinesNumbered
\SetAlgoLined

\Input{este texto} 
%\KwIn{este texto}
\Output{como escribir algoritmos con \LaTeX2e}
%\KwOut{como escribir algoritmos con \LaTeX2e}

inicialización\;
\While{no es el fin del documento}{
	leer actual\;
	\eIf{comprendido}{
		ir a la siguiente sección\;
		la sección actual es esta\;
	}{
		ir al principio de la sección actual\;
	}
}

% Aunque el captión aparece abajo siempre se pone arriba como en tablas y listados
\caption{Cómo escribir algoritmos}\label{alg:como}
\end{algorithm}\DecMargin{1em}

\newpage

\section{Menús, paths y teclas con el paquete \texttt{menukeys}}
Cada vez es más usual que los trabajos en ingeniería exijan el uso de software. Para poder especificar de modo elegante el uso menús, pulsación de teclas y directorios se recomienda el uso del paquete \texttt{menukeys}.\footnote{\url{https://osl.ugr.es/CTAN/macros/latex/contrib/menukeys/menukeys.pdf}} Este paquete nos permite especificar el acceso a un menú, por ejemplo:

\menu{Herramientas > Órdenes > PDFLaTeX}\\

\noindent También un conjunto de teclas. Por ejemplo:
\keys{\ctrl + \shift + T}\\

\noindent O un directorio:
\directory{C:/user/LaTeX/Ejemplos}\\

\noindent Aunque este paquete permite muchas opciones de configuración de los estilos aplicados, no es necesario hacerlo para obtener unos resultados muy elegantes.
















Descripción esgrima

La esgrima es un deporte de estrategia en el cuál tendrás que analizar a tu
 oponente a la vez que te defiendes de sus acometidas. A la vez, tendrás que
 disfrazar tus ataques con otros para que el oponente no sea capaz de analizar
 tus movimientos. Es por esto por lo que la mayoría de practicantes lo denominan
 el ajedrez en movimiento puesto que todo son ataques, por un franco u otro,
 pequeñas batallas que te llevarán a ganar la guerra al final. Jugar con la mente
 del rival y calmar la tuya para tener superioridad táctica.

Por supuesto que el físico influye en este deporte, no deja de ser un deporte de contacto
 en el cual las cualidades físicas (fuerza, agilidad, rapidez, coordinación, reflejos, etc)
 son un factor mas a tener en cuenta, pero esto no serán mas que componentes de una ecuación
 la cual nos dará la victoria.

Al ser un deporte minoritario, que alguna vez hemos visto en la televisión cuando hay
 olimpiadas y nos detenemos un momento a verlo porque hay espadas y vemos a dos personas
 luchando con ellas como si de piratas o de mosqueteros se tratasen, pasemos a explicar por encima
 en que consiste un asalto de esgrima. Un asalto de esgrima es un enfrentamiento entre dos
 oponentes los cuales tienen que llegar al límite de tocados antes que el rival o estar por encima
 de este una vez haya terminado el tiempo de asalto. Dependiendo de la modalidad y categoría
 variaran estos tiempos y límite de tocados. ¿Pero que es un tocado? Un tocado no es mas que
 un punto a tu favor, el cual se puede conseguir tocando al rival o mediante sanciones del rival.
 Un ejemplo de sanciones podría ser un comportamiento antideportivo, salirse de la pista por el fondo,
 dar varias veces la espalda, perder el tiempo repetidas veces mientras vas perdiendo, etc.

Como la mayoría de artes marciales no es mas que un esquema táctico en el cuál tendremos unas variables
 de entrada mediante las cuales determinaremos una salida. ¿Pero como es posible que un deporte de contacto
 se base en una serie de entradas y salidas? Bien, un ejemplo muy básico es el siguiente: ante una acción ofensiva
 directa hacia la parte superior del cuerpo lo lógico es cubrirse esta parte. Aquí es donde entra en juego
 la estrategia de cada componente del combate. Si el atacante sabe que tu reacción ante una amenaza arriba
 será cubrirte esa zona, el amagará con un ataque falso (finta) a una parte del cuerpo y sobre tu acción
 defensiva para evitar esta acometida aprovechará para atacar otra zona que dejaste descubierta por
 defender la primera acción. Por otro lado, el defensor puede analizar al rival y saber que el primer ataque
 no será el verdadero, si no que será una preparación para atacar sobre otra zona, de este modo anticiparse
 y atacar sobre esta preparación o amagar con defenderse sobre la primera zona para después cubrirse la segunda
 y contra-atacar.

Una vez tenemos unas nociones básicas sobre como funciona un esquema táctico en general sobre cualquier
 disciplina de arte marcial o deporte de contacto, pasemos a hablar de la esgrima en concreto.
 Hay tres disciplinas dentro de este deporte: sable, florete y espada. Siendo la primera una modalidad
 en la que se puede tocar con cualquier parte de la hoja, mientras que en las dos últimas son armas de
 estoque, es decir, solo vale tocar con la punta. Puesto que cada modalidad tiene unas normas y la espada
 es la mas practicada y mas sencilla de todas, nos centraremos en ella. En la modalidad de espada se puede
 tocar en cualquier parte del cuerpo, esto incluye desde el pecho, hasta la suela de la zapatilla, pasando
 por la espalda o cualquier lugar que se nos ocurra. Con la única excepción de la nuca, puesto que es la única
 zona en la que no hay protección, para ello hay normas evitando que des la espalda y expongas esta zona
 tan delicada.

% https://es.m.wikipedia.org/wiki/Archivo:Fencing_epee_valid_surfaces.svg
\begin{figure}[htb]
	\centering
	\includegraphics[width=0.4\linewidth]{blancoValido}
	\caption[Blanco válido espada]{Blanco válido espada}
	\label{fig:blancoValido}
\end{figure}

Como hemos mencionado antes es un arma de estoque, por lo que el mecanismo de activación estará en la punta
 y será mediante un botón, el cuál al presionarse sobre un blanco válido cerrará un circuito electrico cuyo
 objetivo es señalizar el tocado. A partir de este momento el rival tiene un breve periodo de tiempo, 0.4 segundos,
 para realizar un tocado sobre el rival y que haya un tocado doble. Pasado este tiempo el circuito se bloqueará
 y solo habrá un tocado válido. A pesar de que haya un tocado doble no quiere decir que siempre sean válidos
 ambos tocados. Mediante las normas se dictaminará si los dos lo son, solo uno o ninguno de ellos lo es.
 Un ejemplo podría ser que uno de los dos tiradores se encuentre fuera de la pista, lo cual anularía su tocado.
 Como se han podido dar cuenta, hemos hablado de tiempo, por lo que otro componente a tener en cuenta es ser mas
 rápido que el rival, esto habrá que tenerlo en cuenta en nuestro esquema táctico para poder decidir una acción en
 la cual, aunque nos toquen, nosotros lo hagamos con suficiente antelación al rival de modo que su tocado no
 sea válido. Tal y como se habló antes los asaltos tienen un límite de tiempo y un límite de tocados,
 este será otro factor a tener en cuenta en nuestro esquema táctico sobre como plantear el asalto.
 Puede que a veces nos interese llevar un asalto hasta el final del tiempo desgastando físicamente al
 rival para aprovechar esta superioridad al final. Otras veces quizás nos interese lo contrario,
 acabar con el asalto cuanto antes para evitar dejar al contrario pensar. Puede que otras veces nos
 interese alargar el asalto al mayor número de tocados posibles puesto que tengamos mayor
 repertorio que nuestro oponente, mientras que en el caso contrario, si tenemos pocas acciones nos interesará
 hacer el menor número de tocados. También habrá que tener en cuenta el marcador y cuanta distancia hay
 hasta el final del combate, si al rival le falta un tocado para ganar, mientras que a nosotros nos faltan
 tres, no nos interesa que haya un tocado doble puesto que el ganaría. Estas son algunas de las variables
 entran dentro de la formula para plantear nuestra táctica en un asalto de esgrima.

Una vez que ya sabemos como funciona un asalto de esgrima podemos hablar sobre como funciona una competición
 de esgrima. Primero hablaremos de las individuales y mas tarde de los equipos. Se explicará el funcionamiento
 de una competición estandar, lo cual puede variar en función de la categoría y tipo de competición.
 En cuanto a las competiciones individuales primero se disputa una fase de grupos, la cual se denomina
 \textit{poule} en la cual se dividen a todos los tiradores en poules (grupos) de seis o siete tiradores en función
 del número de participantes que haya. Siendo siete el número ideal y dejando las de seis en caso de que
 no haya número suficiente de tiradores. Estas poules se hacen en función del ranking de los tiradores inscritos
 a la competición, de manera que estén lo mas equilibradas posibles. Una vez organizadas los poules se da
 comienzo a ellas. En ellas se enfrentan todos los tiradores entre ellos, empezando los que sean del mismo
 país y club, para evitar favoritismos mas adelante. Estos enfrentamientos serán en un único asalto con un
 límite de cinco tocados y una duración máxima de tres minutos. El primero que llegue al límite con diferencia
 de un tocado o quien tenga mayor puntuación al acabar el tiempo será el ganador de este encuentro.

\begin{table}[htb]%
  \centering
  \caption{Ejemplo tabla resultados poule}
  \label{tab:anchura}
  \begin{tabular}{ | l | l | l | l | l | l | l | l | }
    \hline
    & Tirador 1 & Tirador 2 & Tirador 3 & Tirador 4 & Tirador 5 & Tirador 6 & Tirador 7 \\ \hline
    Tirador 1 & x & V & & & & & \\ \hline
    Tirador 2 & 1 & x & & & & & \\ \hline
    Tirador 3 & & & x & & & & \\ \hline
    Tirador 4 & & & & x & & & 2 \\ \hline
    Tirador 5 & & & & & x & & \\ \hline
    Tirador 6 & & & & & & x & \\ \hline
    Tirador 7 & & & & $V_3$ & & & x \\
    \hline
  \end{tabular}
\end{table}

La anterior tabla sería un ejemplo de una tabla de poule en mitad de una competición. Se puede apreciar como se anotan
 las victorias, las derrotas y los resultados de ambas. En caso de obtener una victoria se anotará la puntuación. Una vez
 terminadas todas las poules se obtendrá la clasificación general, obteniendose de la siguiente manera:

\begin{compactenum}
  \item Porcentaje Victorias/Derrotas
  \item Tocados dados - Tocados recibidos
  \item Tocados dados
\end{compactenum}

En caso de empate de todo lo anterior ambos mantendrán el mismo número de serie y se saltará el siguiente. El orden de
 quien estará encima de otro será aleatorio. Una vez obtenida la clasificación general de las poules, se hace un corte
 para eliminar a un porcentaje de los participantes, suele ser un veinte por ciento. Con los tiradores restantes
 de este corte se hace un tablón lo suficientemente grande para acoger a todos los participantes. El número de este
 tablón será una potencia de dos, es decir 2, 4, 8, 16, 32, 64, etc. En caso de no haber participantes suficientes para completar
 el tablón los primeros participantes pasarán exentos de la primera ronda. El resto de la competición es una eliminatoria
 directa en la que el vencedor pasa a la siguiente ronda mientras que el perdedor termina la competición.

\include{./caps/Objetivo}
\chapter{Antecedentes}
\label{cap:Antecedentes}

En esta parte se deben mostrar los conocimientos obtenidos en la búsqueda bibliográfica y no ideas personales del autor; como mucho podrá aportar un comentario de algunas pocas ideas extraídas de las fuentes en que se ha basado. Se articulará esta parte en diversas secciones, que permitan la exposición estructurada y didáctica de los conocimientos de la investigación bibliográfica.
\chapter{Metodología}
\label{cap:Metodologia}

En este capítulo se debe detallar las metodologías empleadas para planificación y desarrollo del trabajo,
así como explicar de modo claro y conciso cómo se han aplicado dichas metodologías.


Obtención de la BBDD.

Ante la nula información almacenada de una forma estructurada se ha tenido que buscar
 diversas maneras de obtener y almacenar la información, estructurarla y tratarla para
 poder sacar conocimiento de ella.

En la página de la federación internacional de esgrima (FIE) se almacenan los resultados
 de las competiciones de los últimos tiempos. De ahí se puede obtener los resultados
 de cada competición tanto como el ranking general, pasando por la fase de poules
 y acabando con los tablones eliminatorios de estos mismos. Viendo toda esta información
 almacenada se decide extraer y almacenar dicha información. Para la extracción se
 utilizarán técnicas de scrapping web mediante la cual se descarga la página y se
 obtiene su contenido para tratarlo. En este caso navegamos por dicha página y una
 vez obtenida la información la guardamos en nuestra BBDD. Puesto que no está del
 todo estructurada tendremos que ir almacenando dicha información en diferentes BBDD
 y después juntarlas.

Descripción de la BBDD.

La primera BBDD que generaremos será aquella en la que guardaremos la información
 básica de los asaltos. Para ello guardaremos el ID de la competición, el número
 de tablón en el que se efectuó el asalto, ID del primer tirador, ID del segundo
 tirador, tocados obtenidos por el segundo primer tirador y tocados obtenidos por
 el segundo tirador.

\begin{table}[htb]%
  \centering
  \caption{Estructura BBDD asaltos inicial}
  \label{tab:anchura}
  \begin{tabular}{ | l | l | l | l | l | l | }
    \hline
    Nombre de Campo & Tipo de campo & Ejemplo \\ \hline
    CompetitionID & Texto & 2019-64 \\ \hline
    Tableu & Entero & 32 \\ \hline
    Competitor1 & Texto & /fencers/Anna-KOROLEVA-40351/ \\ \hline
    Competitor2 & Texto & /fencers/Kira-KESZEI-49034/ \\ \hline
    ResultCompetitor1 & Texto & V/15 \\ \hline
    ResultCompetitor2 & Texto & D/3 \\
    \hline
  \end{tabular}
\end{table}

El siguiente paso que tendremos que dar será obtener la información de los tiradores
 para ello ser visitará la página correspondiente. Un ejemplo de página que almacena
 la información de un tirador sería el siguiente http://fie.org/es/fencers/Mario-PERSU-31870
 como se puede observar el final de la URL es el mismo que el identificador almacenado en la
 anterior tabla. De modo que explorando la anterior BBDD podremos visitar las páginas de cada
 tirador almacenado y de ese modo generar una nueva BBDD con toda la información de cada uno de ellos.
 En dicha BBDD tendremos su identificador, edad, ranking (si lo tuvieran), nacionalidad, mano usada
 y arma.

\begin{table}[htb]%
  \centering
  \caption{Estructura BBDD tiradores}
  \label{tab:anchura}
  \begin{tabular}{ | l | l | l | l | l | l | }
    \hline
    Nombre de Campo & Requerido & Tipo de campo & Ejemplo \\ \hline
    competitorID & Si & Texto & ADRIANA-MILANO-36467 \\ \hline
    Age & Si & Entero & 21 \\ \hline
    FieRanking & No & Entero & 300 \\ \hline
    HandNess & Si & Texto & Right \\ \hline
    Weapon & Si & Texto & Epée \\
    \hline
  \end{tabular}
\end{table}

Una vez obtenidos todos los datos referentes a los tiradores tendremos que cruzar
 las dos tablas mencionadas anteriormente de modo que tengamos toda la información
 en una sola BBDD. Esta última BBDD tendrá la información de la primera, sustituyendo
 las columnas de ID de cada tirador por su registro correspondiente en la anterior tabla.

De modo que la estructura será la siguiente
\begin{table}[htb]%
  \centering
  \caption{Estructura BBDD asaltos final}
  \label{tab:anchura}
  \begin{tabular}{ | l | l | l | l | l | l | }
    \hline
    Nombre de Campo & Tipo de campo & Ejemplo \\ \hline
    ComptetitionID & Texto & 2019-176 \\ \hline
    TABLEU & Entero & 32 \\ \hline
    C1\_ID & Texto & Sergey-KHODOS-13869 \\ \hline
    C1\_RANKING & Entero & 22 \\ \hline
    C1\_NATIONALITY & Texto & RUSSIA \\ \hline
    C1\_HANDNESS & Texto & Right \\ \hline
    C1\_WEAPON & Texto & Epée \\ \hline
    C2\_ID & Texto & Laurin-EGGENSCHWILER-5966 \\ \hline
    C2\_RANKING & Entero & 34 \\ \hline
    C2\_NATIONALITY & Texto & SWITZERLAND \\ \hline
    C2\_HANDNESS & Texto & Right \\ \hline
    C2\_WEAPON & Texto & Epée \\ \hline
    C2\_ID & Texto & Laurin-EGGENSCHWILER-5966 \\ \hline
    C1\_RESULT & Texto & V/15 \\ \hline
    C2\_RESULT & Texto & D/3 \\
    \hline
  \end{tabular}
\end{table}

Modo de empleo.

El modo de empleo de esta BBDD será para el entrenamiento de una red neuronal
 la cual servirá para apoyar al sistema experto. En total hay unos 28.000 registros,
 de los cuales se emplearán entorno al 60 por ciento para entrenar al modelo y un 40
 por ciento para comprobar que el entrenamiento ha sido satisfactorio.

\include{./caps/Resultados}
\chapter{Conclusiones}
\label{cap:Conclusiones}

Breve resumen de lo más destacable del TFG con la solución propuesta y posibles mejoras, ampliaciones o trabajos relacionados que quedan por hacer y que tienen interés para el tema tratado.





%%%%%%%%%%%%%%
% Anexos
\appendix

% Tras este punto los capítulos se numeran con letras.
\ifspanish
\renewcommand{\appendixname}{Anexo}
\fi
\chapter{Anexo A}
\label{cap:AnexoA}

Lista de símbolos, terminología, notaciones, programas, listados, ejemplos, suplementos, fotocopias, dibujos, planos y en general cualquier cosa que considere necesaria el alumno. Así mismo, podrá incluirse en carpetas de plástico o de otro tipo, información en soportes electrónicos. % Apéndice A (opcionales)
%
%............... % Aquí todos los apéndices necesarios


%%%%%%%%%%%%%%
% Bibliografía
\backmatter

\nocite{*} % Comando por si queremos incluir todas las entradas de la bibliografía aunque no estén citadas en el texto pral.(NO ESTÁ PERMITIDO POR LA NORMATIVA DEL ALGUNOS CENTROS (p.ej. ESI)-> Todas las referencias deben estar citadas en el texto)

\ifspanish
\renewcommand{\bibname}{Bibliografía}
\fi
\addcontentsline{toc}{chapter}{\bibname} % Añade la bibliografía al Índice de contenidos.
\printbibliography



%%%%%%%%%%%%%%
% Índice alfabético (opcional)
% OPCIONAL: (Consejo) Incluir los comandos mientras se escribe cada capítulo ya que hacerlo al final resulta tedioso.
\cleardoublepage
\ifspanish
%\renewcommand{\indexname}{Índice alfabético} %Cambio de nombre de sección
\fi
\addcontentsline{toc}{chapter}{\indexname} % Añade al Índice de contenidos.
\printindex  % Facilitado por makeidx (opcional, si no se usa no se imprime)
\end{document}