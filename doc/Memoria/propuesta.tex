\chapter{Propuesta}
\label{cap:Propuesta}

\section{Sistema experto}
En este apartado se abordarán los pasos a seguir para desarrollar un sistema experto.
 En la figura se puede observar el ciclo de vida de creación y mantenimiento de un software
 de este tipo:

\begin{figure}[htb]
  \centering
    \includegraphics[width=0.4\linewidth]{DesarrolloSE}
  \caption[Desarrollo SE]{Desarrollo SE}
  \label{fig:Desarrollo Sistema Experto}
\end{figure}

\subsection{Características del sistema experto}
%inicio pendiente de revision palabras
Cuando se desea implementar un sistema experto en primer lugar se debe definir el
 alcance y los límites de dicho sistema, se deben indicar que aspectos cubre el
 sistema experto y que aspectos quedan fuera del ámbito del sistema.

Define las características del problema. Se pretende determinar la naturaleza
 del problema y los objetivos precisos que indique exactamente cómo se espera
 que el sistema experto contribuya a la solución de los problemas. Existirá
 una interacción entre experto e ingeniero. Cuando el experto en el dominio
 muestre distintos casos, el ingeniero del conocimiento desarrolla una primera descripción
 del problema. Normalmente el experto no esta de acuerdo con ella, o mejor dicho,
 no siente que representa el problema en su totalidad, entonces el ingeniero reformulará
 la descripción. Esta actividad prosigue hasta que los dos estén de acuerdo en la
 descripción.

El método del test de SLAGEL se define en el Anexo I.
%fin pendiente de revision palabras

\subsection{Estudio de viabilidad}
%inicio pendiente de revision palabras
Teniendo claro el alcance del sistema experto debe estudiarse si la implementación
 del mismo es posible desde el punto de vista de la computación. Para esto se
 realiza un estudio de viabilidad, en este caso se utiliza un estudio de viabilidad basado
 en el test de SLAGEL. Este estudio establece sie l proyecto cumple con las siguientes
 características.

\begin{compactitem}
  \item \textbf{Plausible}: Determina si es posible resolver el problema desde el
     punto de vista de la ingeniería del conocimiento.
  \item \textbf{Justificable}: Analiza si está justificado el desarrollo del
     sistema desde la perspectiva d la ingeniería dle conocimiento, se basa en
     temas como la necesidad del sistema y la inversión a realizar.
  \item \textbf{Adecuado}: Establece si el problema a resolver está dentro
     del marco de la ingeniería del conocimiento, existen problemas que son más
     adecuados resolverlos por métodos tradicionales.
  \item \textbf{Éxito}: Determina las probabilidades de éxito del sistema a
     desarrollar, es una estimación
\end{compactitem}

El método del test de SLAGEL se define en el Anexo I.
%fin pendiente de revision palabras


\subsection{Adquisición del conocimiento}
%inicio pendiente de revision palabras

Una de las partes más importantes y a la vez más complicadas de la implementación
 de un sistema experto es obtener el conocimiento, normalmente un experto, y
 volcarlo en el sistema experto. Existen muchas herramientas y métodos para obtener
 este conocimiento entre las cuales está la entrevista, la observación y la creación
 de escenarios.

La adquisición del conocimiento es la principal complicación en el desarrollo de
 Sistemas Expertos. Consiste en que las personas no expertas en el dominio donde
 se va a desarrollar el Sistema Experto extraigan el conocimiento necesario para resolver
 problemas de diversas fuentes. El proceso de adquisición del conocimiento ha seguido
 diferentes etapas que podrían resumirse en las siguientes:

\begin{compactitem}
  \item Primeras reuniones con los expertos y evaluación de la viabilidad del proyecto.
  \item Extración de conocimientos, a partir de la documentación disponible, como
     por ejemplo libros, conferencias, internet, etc.
  \item Deducción de conocimientos a partir de los expertos.
\end{compactitem}

Además se logra la familiarización del Ingeniero del Conocimiento en el contexto en
 el que se va a trabajar. Se busca en las primeras reuniones describir conocimientos
 generales, así como afianzarse con la terminología.

La estructura de las entrevistas se define en el anexo III.

Una vez obtenido el conocimiento hay que designar estructuras para organizar el
 el conocimiento. Después de haber determinado el problema en toda su magnitud,
 sin haberse referido a técnicas de programación o a indagar solo en los métodos
 que son exitosos en inteligencia artificial, es en esta etapa donde el ingeniero del
 conocimiento selecciona las estructuras apropiadas a este sistema experto en particular.
 Es decir, que dan solución total o parcial al problema analizado en las etapas precedentes.
 Una de las responsabilidades principales del ingeniero del conocimiento es analizar
 situaciones tipo y a partir de ellas extraer las reglas que describen el conocimiento del
 experto en el dominio.

%fin pendiente de revision palabras

\subsection{Implantación}
%inicio pendiente de revision palabras

Desarrolla la transformación de los conocimientos representados en el modelo formal
 en un modelo computable.

Elaboración de las reglas que incorporen el conocimiento. Se peretende en esta ocasión
 usar las herramientas y técnicas predeterminadas para implementar una primera versión
 o prototipo del sistema. Este prototipo esta destinado a evaluar los progresos que se
 van haciendo, y por ende, retornar a etapas anteriores si es necesario.

Una vez que el sistema prototipo se ha perfeccionado lo suficiente para ser ejecutado,
 el sistema experto estará listo para ser probado.

%fin pendiente de revision palabras

\subsection{Evaluación y pruebas}
%inicio pendiente de revision palabras

Establece el grado de experiencia alcanzado por el sistema. De manera tal que expertos
 en el área que han o no partilcipado en el desarrollo del proyecto se comprometen a
 evaluar el desempeño del sistema, tratando de vislumbrar la calidad de asistencia que
 brinda el Sistema Experto ante diferentes casos de problemas a resolver por software.
 También se evalúa la amplitud y generalidad de marcos compuestos que posee el repositorio
 y cómo el sistema guía su uso.

Se considera que el sistema experto está terminado cuando realiza trabajos a nivel
 del especialista. Entonces, el proceso de prueba no esta listo hasta que las soluciones
 propuestas por el sistema seantan válidas como las propuestas por el experto humano.
%fin pendiente de revision palabras

\subsection{Motor de inferencia}
%inicio pendiente de revision palabras
El sistema experto que se va a desarrollar siguiendo la metodología anterior va
 a funcionar sobre un motor de inferencia desarrollado para tal fin, el motor
 de inferencia se basa en el motor implementado para CLIPS y a continuación se van
 a detallar las características de dicho motor.

\subsubsection{Algoritmo de selección de reglas aplicables en CLIPS}
\begin{compactitem}
  \item Elegir la regla aplicable con máxima prioridad.
  \item Elegir la regla según estrategia de resolución de conflictos.
  \item Elegir de forma arbitraria.
\end{compactitem}

Estas son las estrategias definidas en el motor de inferencia de CLIPS para
 la selección de reglas aplicables o activas son varias:

\begin{compactitem}
  \item \textbf{Depth Strategy (estrategia por defecto)}. Una activación que contiene el hecho
    más reciente se sitúa por encima de las activaciones con igual o mayor antigüedad.
 \item \textbf{Breadth Strategy}. Una activación que contiene el hecho más reciente se
    sitúa por debajo de las activaciones con igual o mayor antigüedad.
 \item \textbf{Complexity Strategy}. Las nuevas activaciones se sitúan por encima de las
    activaciones con igual o menor especificidad (no de comparaciones que han de
    realizarse en el antecedente una la regla).
 \item \textbf{Simplicity Strategy}. Las nuevas activaciones se sitúan por debajo de las
    activaciones con igual o mayor especificidad.
 \item \textbf{LEX Strategy}. Se ordenan los time-tag en orden decreciente y se comparan
    uno a uno, hasta encontrar uno mayor que otro, en caso de que no haya el mismo
    número de time-tag se añaden ceros al final.
 \item \textbf{MEA Strategy}. Parecido a LEX, pero mirando sólo el primer patrón que
    equipara en la regla.
 \item \textbf{Random Strategy}. A cada activación se le asigna un número aleatorio para
    determinar su orden en la agenda.
\end{compactitem}

\subsubsection{Definición de prioridades en CLIPS}
\begin{compactitem}
  \item Asignar un valor de prioridad (entero positivo) a cada regla.
  \item En CLIPS se definen como propiedades de reglas.
  \item Se recomienda minimizar el uso de prioridades de reglas.
\end{compactitem}

%fin pendiente de revision palabras

