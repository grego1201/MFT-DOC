En este capítulo se hablará sobre como se ha desarrollado el proyecto. Para ello primero hablaremos
de la metodología utilizada para después pasar a explicar cuál fue el proceso de desarrollo del sistema.

\section{Metodología}

Para poder escoger una buena metodología tenemos que saber de los recursos que disponemos además
de cuales son nuestros objetivos (funciones a desarrollar en que tiempo). Previamente se hablo
de los objetivos pero un resumen de ellos sería desarrollar una aplicación, con la mayor
accesibilidad posible para ayudar al mundo de la esgrima. Para ello se llegó a la conclusión de que
se desarrollaría un sistema de apoyo a la decisión el cuál tendría una interfaz web, de este modo
se podría acceder a ella desde cualquier sitio con Internet. Este proyecto ha de desarrollarse
en un periodo de tiempo de unas 300 horas aproximadas por lo que no podremos llegar a
desarrollarlo por completo, de modo que este será un factor importante a la hora de elegir la metodología.
La que escojamos deberá favorecer el trabajo en funcionalidades completas, de modo que cada vez
que se empiece una, se deberá acabar.

Otra cosa a tener en cuenta son los recursos de los que se dispone. En este caso se dispone de
un desarrollador el cual hará a su vez de Ingeniero del Conocimiento. También se dispone de un
experto en la materia, cuyas horas no serán contabilizadas. Además dichas personas no están a tiempo
completo dedicadas al proyecto, solo podrán dedicarle tiempo de forma ocasional. Esto hace que sea
mas complicado el desarrollo del mismo, por lo que no se podrán tener varias funcionalidades abiertas
sin acabar.

Además sabemos cuál es nuestro punto de partida, al igual que nuestra meta, pero el camino es
en su mayoría incierto. Esto hará que sea realmente difícil definir una serie de tareas, con
pasos a seguir, las cuales tengan una duración estimada fiable.

Por todo lo mencionado anteriormente deberemos escoger una metodología de desarrollo ágil, la cual
nos permita adaptarnos a los posibles cambios e inconvenientes que vayan surgiendo en el propio
desarrollo. También deberemos escoger una metodología incremental, la cuál nos permitirá aumentar
los objetivos de la aplicación en cualquier momento del desarrollo.
