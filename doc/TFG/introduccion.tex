\chapter{Introducción y objetivos}
\label{cap: Introducción y objetivos}

Desde el inicio de los tiempos los humanos hemos competido los unos contra los otros,
siendo este un beneficio para la sociedad, ya que la competencia promueve a la mejora
continua de aquel que quiere llegar a ser el mejor. Esto hace que evolucionen métodos
para desarrollarse en todos los sentidos. En el mundo del deporte hay varios ejemplos
como se ha utilizado la tecnología para cambiar el resultado de un enfrentamiento.
Hace 20 años en el fútbol no se utilizaba la tecnología en el descanso para analizar
las jugadas del rival, las acciones que llevaron a cabo y aprovecharse de ellas.
Actualmente hay varias cámaras grabando cada acción y movimiento para estudiarlos
en el transcurso de la primera parte y así explicárselas a los jugadores en el descanso.
De este modo consiguen una visión externa de que esta pasando en el partido y
poder adaptarse a la situación, consiguiendo una ventaja táctica sobre el rival.

Este mismo método se transcribe a los entrenamientos, cuando los porteros estudian
a los lanzadores de penaltis, para saber que movimientos corporales hacen en cada
lanzamiento y hacia donde van sus disparos con cada uno de estos. De esta manera
aquel portero que parece \textit{adivino} lo que realmente hizo fue un estudio
de sus rivales.

De igual modo la tecnología se empezó a usar para recopilar datos y después analizar
los mismos para comprobar que métodos de entrenamiento resultaban mas efectivos
en cada deportista. Pudiendo desarrollar después diferentes métodos para cada
uno de ellos en diferentes momentos de la temporada, haciendo que estos estén
al máximo nivel requerido en cada momento.

Uno de los grandes debates en el deporte es la iniciación de una persona
en el mismo. Casi todo el mundo coincide en que si empiezan en la escuela base se pueden
asentar de una manera mas sencilla las bases y cimientos del mismo. Esto es debido
a la gran capacidad de aprendizaje de los niños. También se debe a que la tener
un mayor tiempo para poder aprender, es decir tienen toda su vida por delante,
no es necesario enfrentarles a problemas complejos desde el inicio, ya que
sus rivales tendrán la misma experiencia que ellos y no poseerán de una
ventaja en cuanto a experiencia se refiere. Sin embargo cuando el que se adentra
en un deporte competitivo, con la idea de enfrentarse a rivales y conseguir resultados
en el menor tiempo posible al principio tendrá una barrera muy complicada. Dicha barrera
es la experiencia y tiempo practicando. Este tiempo le dará una ventaja táctica y unas tablas
que será muy difícil de superar, ya que en la mayoría de disciplinas la técnica
te hace superar con creces el físico. Es por esto que aquellas personas que
quieran enfrenterase desde el principio a sus rivales contarán con una gran
desventaja y es el desconocimiento de muchas situaciones. Estas situaciones
te hacen aprender de ellas, pudiendo ver que funciona y que no. Ya sea por
ensayo de prueba y error o porque tengas alguien que te guíe en el camino
y te haga ver, ayudándote en todo lo posible, para que puedas entender
dichas situaciones.

\section{Objetivos}

El principal objetivo de este TFG es contribuir al mundo del deporte, en concreto al
 deporte de la esgrima. En el comienzo de este deporte hay una gran curva de aprendizaje
 en cuanto al esquema táctico se refiere puesto que al ser un deporte minoritario los
 recursos que se le dedican son menores por lo que dificulta la expansión de conocimiento
 y por ende, la adquisición de este mismo tanto a personas que ya lo practican como aquellas
 que acaban de empezar. Se quiere reducir la inclinación de dicha curva de aprendizaje
 en el momento que tienes que aprender por ti mismo y necesitas ayuda de los demás
 para saber que acciones son las correctas y porque están mal tomadas algunas decisiones,
 al menos, hasta que te puedes valer por ti mismo como tirador que es capaz de identificar
 las acciones que están ocurriendo y analizar cuales son las mejores decisiones para contrarrestarlas.

Para ello se pretende desarrollar una aplicación la cual sea capaz de llevar a
 cabo una toma de decisiones con una serie de entradas, las cuales serán aquellas
 relacionadas con el entorno de un asalto de esgrima, como son las características
 de los tiradores, como se está desarrollando el asalto, etc.

Esta aplicación será desarrollada llevando a cabo una labor de Ingeniería de Conocimiento
 junto con una extracción y análisis de datos para describir el problema, extraer
 el conocimiento de expertos, conceptualizar, formalizar e implementar dicho conocimiento
 de manera entendible para los usuarios de dicha aplicación. Dicha aplicación será un SBC
 el cual aglutine todo el conocimiento de los expertos en su conjunto, además del análisis
 de datos. Dicho sistema tendrá el objetivo darle una respuesta a un tirador de tal forma
 que este tenga un punto de vista más para tomar sus decisiones, de tal modo que le sea mas
 fácil alcanzar la victoria. Además de ayudarle a anteponerse a su rival, también servirá
 como entrenamiento y salir de dudas cuando se quiera mejorar y adquirir conocimiento.

Esto nos lleva a la conclusión de que para alcanzar el objetivo de este TFG tendremos
 que diseñar un sistema de apoyo a la toma de decisión, con acceso mediante una aplicación
 web para darle accesibilidad al programa desde cualquier sitio.

El alcance de este proyecto está basado en los recursos disponibles para realizarlo, tanto personas como tiempo.
 Varios autores han escrito libros para plasmar su conocimiento sobre este deporte,
 ya sea como plantear la gestión de un club, como preparar a los tiradores para competiciones,
 como iniciarlos, etc. Este último caso es el de Elain Cheris hablando sobre los fundamentos
 básicos de la esgrima en las modalidades de florete y espada ya que ambas comparten
 las bases. Este libro Manual de esgrima consta de 160 páginas en el que se habla sobre
 el primer año de aprendizaje de una persona que se inicia en el deporte. Para adquirir
 este conocimiento se requiere de muchas horas de trabajo y entrevistas con profesionales
 por lo que automáticamente descartamos la modalidad de sable, ya que hay poco conocimiento
 reutilizable.

Debido a los motivos expuestos anteriormente la modalidad de sable se dejará para un futuro
 a modo de ampliación. Respecto a la modalidad de florete es cierto que comparten
 las bases pero las técnicas específicas y el conocimiento es totalmente distinto,
 ya que las propias reglas difieren entre ambas modalidades en algunos aspectos,
 por lo que se podrían utilizar partes del desarrollo pero toda la adquisición de
 conocimiento, desarrollo del sistema experto habría que realizarlo partiendo de cero.
 Para ambas modalidades hay que sumar que conseguir expertos resulta de gran dificultad
 actualmente, cosa que en un futuro lejano, tres años, espero solventar. Todo esto ha
 llevado a los objetivos expuestos anteriormente.

\section{Estructura del trabajo}

En este apartado se pretende dar al lector una idea general del contenido del presente
documento, donde se explicará de forma breve cada uno de los capítulos del mismo, de modo
que su accesibilidad a información de su interés sea lo más sencilla posible.

\subsubsection{Capítulo 1: Introducción y objetivos}
En este capítulo se establece una visión global del proyecto. También se marcará
que queremos realizar dentro del proyecto y que no.

\subsubsection{Capítulo 2: Estado del arte}
Aquí se realiza un estudio bibliográfico de los temas a tratar en el proyecto.

\subsubsection{Capítulo 3: Test de viabilidad}
Estudio exhaustivo de la viabilidad del proyecto.

\subsubsection{Capítulo 4: Propuesta}
Capítulo en el que se trata como será el proceso de desarrollo del proyecto, visto
desde un punto de vista metodológico.

\subsubsection{Capítulo 5: Desarrollo}
En este capítulo se hablará sobre el desarrollo llevado a cabo en el proyecto.
Incluyendo los problemas que surgieron por el camino y como se resolvieron.

\subsubsection{Capítulo 6: Evaluación y resultados}
Capítulo en el que se evaluará el desarrollo del proyecto mediante diferentes
técnicas.

\subsubsection{Capítulo 7: Conclusiones}
Último capítulo en el que se recapacitará sobre el proyecto, todo el proceso llevado
a cabo con los resultados obtenidos ya en mano. También se tratará el trabajo
futuro que le queda.
