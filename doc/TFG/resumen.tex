\chapter{Resumen}

La curva de aprendizaje de cualquier deporte es muy inclinada y por lo general
se necesita de alguien que te guié para hacer que la inclinación de esta se vea
reducida. Muchos deportes se basan en la toma de decisiones para después ejecutarla
y es la mezcla de ambas quien decide el ganador en un enfrentamiento. Esta decisión
puede ser tomada por el deportista o su entrenador si se dispone del mismo.
En deportes minoritarios no siempre se dispone de un entrenador que te ayude
en un enfrentamiento, por lo que si a esto le sumas que eres novato hará
que la entrada a la competición sea costosa e incluso frustrante. Esgrima
es un deporte con estas características: hay que tomar gran cantidad de decisiones
muy complejas para luego llevarlas a cabo. También hay que adaptarse en poco tiempo
ya que estas deberán variar a lo largo del enfrentamiento.

Esta entrada se podría ver ayudada de una aplicación que ayude a la toma
de decisiones ya que es lo mas complejo al inicio. Dicha aplicación será elaborada,
en parte, durante el desarrollo de este TFG. En parte porque ser desarrollará un
prototipo funcional debido a la complejidad de desarrollar por completo la aplicación.
Para ello se utilizarán diversas técnicas de aprendizaje automático junto a sistemas
basados en el conocimiento. Dicha aplicación conseguirá ayudar a los que
se inicien en la esgrima deportiva ayudandoles a tomar mejores decisiones y comprendiendo
el por qué de estas.

Para ello se contará con la ayuda de un gran experto en el sector. Este ayudará
en la parte del sistema basado en el conocimiento, el cuál plasmará todo el conocimiento
obtenido con la experiencia de situaciones vividas junto a los años en el sector. Por
otro lado se generó una base de datos extrayendolos y estructurándolos ya que no
existía nada parecido.

\chapter{Abstract}

English version of the previous page.
