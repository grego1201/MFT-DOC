\subsection{Base de datos y Web Scrapping}
Ante la inexistencia de una base de datos con datos relevantes para el prósito de este proyecto
 se decide generar una propia como ya se hizo previamente en el experimento mencionado anteriormente
 del DSS para los entrenamientos. En su caso los datos fueron generados mediante varios sensores y
 datos de entrenamientos apuntados por ellos mismos. En nuestro caso generaremos nuestra propia
 base de datos mediante la obtención de estos mismo a través de internet. Para ello consultaremos
 las pertinentes páginas y se usará un proceso de automatización de recogida de datos como es
 Web Scrapping.

Web Scrapping es un proceso de automatización para la recolección de datos de la web. Las páginas
 más accesibles para ser scrapeadas serán aquellas cuyo lenguaje se ejecute en el cliente como
 es el caso de HTML o XML, es decir, aquellos en el que se descargue el código fuente y lo interprete
 el navegador. Para ello primero se descarga el código fuente de la página de la que queremos
 obtener los datos para después navegar por ella y obtener la información que nos sea necesaria.
 Además, podremos seguir navegando, visitando otros enlaces de la propia página e interactuar
 con ella como si fuera un humano quien la visitara desde su computador. Un ejemplo se puede
 ver en la aplicación que hicieron para un sistema que te ayudara a encontrar trabajo en el cual
 utilizaban redes bayesianas y scrapeaban los datos que necesitaban para generar la base de datos
 de la cual pudieron generar las redes bayesianas y después entrenarlas.
